\documentclass[11pt,conference]{IEEEtran}
\title{Promiscuous Speculation: When is it inappropriate to flaunt your guesses?}
\author{William Wesley \\
\texttt{wwesley@uccs.edu} \\
University of Colorado \\
Colorado Springs, Colorado}

\usepackage[T1]{fontenc}

\begin{document}

\maketitle

\begin{abstract}
Lalalal...
\end{abstract}

\section{Introduction}

Speculative execution attacks have rightfully garnered much attention in the past few years.
With the publication of the Meltdown\cite{meltdown} and Spectre\cite{spectre} vulnerabilities in 2018 and 2019, respectively, the public became acutely aware of some promiscuous behaviors undertaken by modern processors in the name of performance.
These behaviors lead directly to compromises in system security by violating the abstraction between hardware and software.

However, the performance gains obtained by speculative execution are so great that we refuse to abandon it and instead seek to mitigate the vulnerabilities in a combination of hardware and software patches.
We reduce the fidelity of side-channels to deteriorate the performance of attacks to the point that there is more likely an easier way to compromise the system elsewhere.
We prevent access to sensitive areas of memory by obscuring their location or confining speculative accesses.
We disable speculative execution with ``fences'' that are software instructions that force the processor into taking more care in particularly risky code structures.

In \S\ref{sec:background} we discuss what speculative execution attacks are and what enables them.
In \S\ref{sec:related} we talk specifically about work modelling software to identify ``gadgets,'' which can be used to identify places in software where fences or access confinement are necessary to ensure system security.
Then in \S\ref{sec:future} we identify potential avenues for future research in this area.

\section{Background}\label{sec:background}

Modern general purpose processors derive a large part of their performance from ``speculative execution.''
Control-flow-affecting instructions present a problem to performance in that the information required to make a control flow decision may not be available before the decision must be made.
Waiting for that information will result in delays and degradation of system performance.
However, if the system can accurately predict\footnote{That is, guess better than a coin flip.} the decision, then it can begin execution of the instructions in the correct flow of control before that decision can be made.
Any uneasiness with this proposal is calmed with the promise that if the prediction is incorrect, the system can simply ``roll-back'' the effects of the incorrect control flow and the penalty for doing so will not be much more than if the system waited for all the information.
In this way, we optimize for the common case where our predictions are correct and accept some penalty in the rare case that our predictor fails us.

Another tool employed, in many variations, to increase system performance is \emph{caching}.
We will define caching in this paper as any memory structure that is used to avoid repeating previous work.
For example, a request to main memory can be cached so that if the same request is repeated, it can be satisfied from the cache rather than a whole new request to main memory.
Similarly, a translation from virtual memory to physical memory can be saved in a cache so that next time the operating system need not be invoked to perform the translation.
Yet another example is fast memory locations built into larger memory components that store the results of recent requests to the component, such as row buffers in main memory or cache memory on a disk drive.
In all of these cases, temporal and physical locality is exploited to reduce the time required to complete execution of a program by avoiding the performance on all of the ``movements'' that would be strictly required.

But, variations in performance are exactly the kind of behavior that lead to Timing Attacks as described by Kocher over twenty-five years ago\cite{kocher96}.
If the system performance varies with the input of some secret value, that value can potentially be inferred from that variation - that is the variation in time is a ``side-channel'' to obtaining the secret that the software developer may or may not have accounted for in constructing the system.
Caching alone provides this potential side-channel\cite{shepherd2022transient}, but when combined with speculative execution, the risk is amplified.

In a system with cache in place, the results of a computation or other activity are available quicker if those results were recently obtained.
In one approach, an attacker might cause a series of computations to be performed and compare the relative times to identify which computation was recently performed, thus identifying which one was executed by the victim and uncovering potential secrets.
Such an attack would be expensive, but still possible.
With speculative execution, if an attacker can cause the predictor to make an incorrect prediction while influencing what the incorrect control flow instructions do, the attacker can side step controls such as bounds checking or privilege levels in a brief window within which cache states can be affected in a relatively precise manner.

Furthermore, as Shepard, Brookes, and Denz\cite{shepherd2022transient} remind us, memory access timing is not the only side channel that might be present.
Speculative execution may also leak information in terms of varying power consumption or EM radiation.
\emph{Any} side effect changes the state of the system and potentially can be used to exfiltrate sensitive data.

Exact details vary from variant to variant of speculation attack, so for this paper we will summarize the vulnerability as situations in which speculative execution causes changes in state that is not rolled back.
Although this vulnerability is ostensibly a hardware problem since it is the processor making the decision to violate program order and execute software instructions ``on a guess,'' there are software structures that are more susceptible to falling victim to these kinds of attacks and we can mitigate the risk by modifying our code, masking our data, or explicitly instructing the processor to forego speculation.
These code structures are called \emph{gadgets} in the literature.

{\Large stick example gadget from original spectre paper here}

\section{Related Work}\label{sec:related}
In \textit{A Systematic Evaluation of Transient Execution Attacks and Defenses}, Canella et al\cite{canella2019} describe Spectre and Meltdown as \emph{transient} execution attacks, that is they characterized the vulnerabilities as being the result of instructions being executed and then rolled back.
This is slightly broader than our definition in that we've focused specifically on speculative execution, whereas the transient definition would allow for instructions rolled back for some other reason than an incorrect prediction.
For example, instructions following one that creates a fault late in a pipeline may affect system state and not be the fault of a misprediction but just the normal flow of the program.
However, in a sense this is still a form of misprediction on the part of the compiler or software developer, so we chose to use the traditional nomenclature for our paper.
Their systematic evaluation of Spectre and Meltdown style attacks allowed them to formulate a classification system.
They were able to use this system to give more descriptive names to attacks that were currently known and also to identify several new attack variations.

Since Canella et al distinguish between Spectre type and Meltdown type vulnerabilities as being caused by speculation or fault, respectively, it is natural for them to classify the defenses along the same lines.
For Spectre, they give three categories of defense, paraphrased here:
\begin{enumerate}
	\item[\textbf{C1:}] Reducing the fidelity of covert channels, for example by adding jitter to or reducing the precision of timer related functions,
	\item[\textbf{C2:}] Inhibiting speculation when sensitive data might be accessed, for example placing \texttt{fence} instructions around potential gadgets, or
	\item[\textbf{C3:}] Ensuring sensitive data cannot be reached, for example masking values used in offsets can affect limits on the range available to out-of-bounds accesses in a gadget.
\end{enumerate}

For Meltdown, they divide the defenses into the following two categories:
\begin{enumerate}
	\item[\textbf{D1:}] Ensuring sensitive data cannot be reached, for example by storing such data in uncacheable memory or structures that are always sanitised on context switches, or
	\item[\textbf{D2:}] Preventing faults from occurring, for example an access to a protected page from an unprivileged mode might return invalid values instead of triggering a fault.
\end{enumerate}


\cite{mosier2022}.
\cite{cats2022}
\section{Future Work}\label{sec:future}

Canella et al's success in classifying attacks which led to identifying additional variants gives us hope that a similar effort with respect to the defenses might yield additional defenses that might prove useful against future variants.
We believe there is opportunity both in expanding their classification system in the details and consolidating common features across attack variants.
There is significant overlap in their categories C2 and D2 and there are overlaps in C3 and D1.
An overarching trifurcation of approaches, subdivided on perhaps hardware vs software solution, might prove useful in identifying gaps in defense research.
For example, is there value in identifying ways to reduce the accuracy of data leaked to a Meltdown attack?

\section{Conclusion}
Summarize and incite.

Shepard, Brookes, and Denz \cite{shepherd2022transient} mention power consumption and EM radiation as potential side channels.
Could be sad sad how to axiomatic model \emph{all} side effects.
They also brought my attention to the fact that cache by itself presents a side channel with or without speculative execution.
Since caching can affect access speed in either case, a sufficiently precise timer can detect which cache lines have been recently touched and from there make inferences about values.
This is possible because addresses are the same thing as data - one can use data as an address and thereby affect cache state in a manner directly correlated to the data value.

Something about optimize for the common case.
Transient execution exploits the edges.
Can coin flips provide security and an acceptable performance?
Complicated machinery in processors to make predictions, throw predictions out and take an 80\% hit to performance.
But, replace them with highly/suffficiently rng, and make coin-flip predictions - 50\% of the time they're right every time?
Adds a ton of noise to the side channels, how much does it really affect performance from a non-speculative baseline or from the super advanced but exploitable current gen?

Lawl, I don't really want to pursue that, do I?
Would be hilarious, I think.

In Ponce-de-León and Kinder\cite{cats2022}, the reference another work about assume always mispredict and you can render security guarantees.
Could we show a tighter bound by means of coin flips?


Despite being discovered twenty-five years ago\cite{kocher96}, even software developers of secure libraries fail to employ tools to defend against timing attacks\cite{9833713}.


\bibliographystyle{ieeetr}
\bibliography{paper}
\end{document}

